%%%% Small single column format
\documentclass[anonymous=false, %
               format=acmsmall, %
               review=true, %
               screen=true, %
               nonacm=true]{acmart}

\usepackage[ruled]{algorithm2e} 
%\usepackage{parskip}
\usepackage{backnaur}
\usepackage{jlcode}

\urlstyle{tt}
\citestyle{acmauthoryear}

\begin{document}

\title{Soss}
%  \titlenote{This is a titlenote}
%  \subtitle{This is a subtitle}
%  \subtitlenote{Subtitle note}

\author{Chad Scherrer}
\orcid{0000-0002-1490-0304}
\affiliation{%
  \institution{RelationalAI}
  % \department{Department of Brain and Cognitive Sciences}
  %\streetaddress{43 Vassar St}
  %\city{Cambridge}
  %\state{MA}
  %\postcode{02139}
  %\country{USA}
}
\email{chad.scherrer@gmail.com}

\author{Taine Zhao}
%\orcid{1234-5678-9012-3456}
\affiliation{%
  \institution{University of Tsukuba}
  \department{Department of Computer Science}
  %\streetaddress{625 Mt Auburn St #3}
  %\city{Cambridge}
  %\state{MA}
  %\postcode{02138}
  %\country{USA}
}
% \email{apfeffer@cra.com}
%\renewcommand\shortauthors{Mage, M. et al}

\begin{abstract}
Abstract goes here
\end{abstract}

\maketitle

\section{Introduction}

Here's some example code. Lots of dead space on the right, maybe want to try to keep code in a multi-column figure

\begin{lstlisting}[language = Julia]
  using Plots
  
  x = -3.0:0.01:3.0
  y = rand(length(x))
  plot(x, y)
\end{lstlisting}
  

\section{End Matter}

Identification of funding sources and other support, and thanks to
individuals and groups that assisted in the research and the
preparation of the work should be included in an acknowledgment
section, which is placed just before the reference section in your
document. This section has a special environment:
\begin{verbatim}
  \begin{acks}
  ...
  \end{acks}
\end{verbatim}
This ensure that information contained therein can be more easily collected during the article metadata extraction phase, and ensures consistency in the spelling of the section heading. An example is below.

The \verb|\appendix| command is used to start the appendix sections in the document, which are indexed with letters rather than numbers. This command should be included after  the \verb|\bibliography| command, which creates the references section. 

\section{Model syntax}


\begin{figure}[!t]
  \centering
  %\fbox{\rule[-.5cm]{0cm}{\linedwid} \rule[-.5cm]{4cm}{0cm}}
  % using the backnaur package
\begin{bnf*}
  \bnfprod{model}{\bnfts{@model} \bnfsp \bnfpn{args} \bnfsp \bnfts{begin} \bnfsp \bnfpn{statements} \bnfsp \bnfpn{retn} \bnfsp \bnfts{end}}\\
  \bnfprod{args}{\bnfes \bnfor \bnfpn{Symbol} \bnfor \bnfpn{Symbol} \bnfts{,} \bnfpn{args}} \\
  \bnfprod{statements}{\bnfpn{statement} \bnfor \bnfpn{statement} \bnfts{\textbackslash n} \bnfpn{statements}} \\ 
  \bnfprod{statement}{\bnfpn{assign} \bnfor \bnfpn{sample}} \\
  \bnfprod{assign}{\bnfpn{Symbol} \bnfsp \bnfts{=} \bnfsp \bnfpn{Expr}} \\
  \bnfprod{sample}{\bnfpn{Symbol} \bnfsp \bnfts{\textasciitilde}} \bnfsp \bnfpn{Measure} \\
  \bnfprod{retn}{\bnfts{return} \bnfsp \bnfpn{Expr}} \\
  \bnfprod{Symbol}{\bnftd{Julia Symbol}} \\ 
  \bnfprod{Expr}{\bnftd{Julia Expr}} \\
  \bnfprod{Measure}{\bnftd{Probability measure (see text)}}
\end{bnf*}
  \caption{Backus-Naur Form representation for a user-specified model.}
  \label{fig:bnf}
  \Description{Placeholder figure.}
\end{figure}

\section{Implementation}



\section{Performance}



\begin{acks}
We would like to acknowledge...
\end{acks}

\bibliographystyle{acm-reference-format}
\bibliography{probprog-2020-instructions}

\appendix

\section{Appendices}

Authors may provide appendices to accompany their extended abstract submissions. Please note however that reviewers will not be required to comment on material in these appendices.
\end{document}
